% In total max. 1 Page!
\AIRstudentthesisAbstract{
    Object contour reconstruction is critical in robotics for tasks such as navigation and recognition.
    Whisker sensors offer a promising tactile modality due to their high spatial resolution, robustness under varying conditions, and low computational requirements.
    However, existing approaches typically fail to reconstruct objects with sharp corners, as the whisker detaches from the object, and do not allow for contour reconstruction in confined spaces.
    This work introduces a development platform integrating multiple magnetically transduced whisker sensors mounted on a robotic arm for active control.
    For single-whisker configurations, we propose an enhanced contour-following strategy paired with an object retrieval policy to manage unavoidable whisker detachments at sharp corners and accurately capture their contours.
    A tunneling policy is introduced for multi-whisker setups, and all control policies are combined into a single finite state machine that dynamically selects the active policy and determines which whiskers are relied on.
    Simulation results demonstrate that our method effectively reconstructs contours with sharp angles with sub-millimeter accuracy and maintains a centered trajectory in confined passages.
    Finally, we present a comprehensive framework for real-world experiments, including data collection, preprocessing, evaluation, storage, and visualization.
}{
    Die Rekonstruktion von Objektkonturen ist essenziell in der Robotik für Aufgaben wie Navigation und Objekterkennung.
    Whisker-Sensoren bieten aufgrund ihrer hohen räumlichen Auflösung, Robustheit unter wechselnden Bedingungen und geringen Rechenanforderungen eine vielversprechende taktile Modalität.
    Bestehende Ansätze scheitern jedoch häufig an der Rekonstruktion von Objekten mit scharfen Ecken, da sich die Whisker hierbei vom Objekt lösen, und ermöglichen keine Konturrekonstruktion in engen Räumen.
    Diese Arbeit stellt eine Entwicklungsplattform vor, die mehrere magnetisch transduzierte Whisker-Sensoren integriert, welche aktiv durch einen Roboterarm gesteuert werden.
    Für Ein-Whisker-Konfigurationen schlagen wir eine verbesserte Konturverfolgungsstrategie vor, kombiniert mit einer Strategie der Objektwiederaufnahme, um unvermeidliche Whisker-Ablösungen an scharfen Ecken zu bewältigen und deren Konturen präzise zu erfassen.
    Für Multi-Whisker-Konfigurationen wird eine Tunneling-Strategie eingeführt, und alle Steuerstrategien werden in einer einzigen Zustandsmaschine kombiniert, die dynamisch die aktive Strategie auswählt und bestimmt, auf welche Whisker vertraut wird.
    Simulationsergebnisse zeigen, dass die vorgestellte Methode Konturen mit scharfen Winkeln im Submillimeterbereich genau rekonstruiert und eine mittige Trajektorie in engen Passagen aufrechterhält.
    Abschließend präsentieren wir ein umfassendes Framework für reale Experimente, inklusive Datenerfassung, Vorverarbeitung, Auswertung, Speicherung und Visualisierung.
}
