% In total max. 1 Page!
\AIRstudentthesisAbstract{
    Object contour reconstruction is critical in robotics for tasks such as recognition, manipulation, and navigation.
    Whisker sensors offer a promising tactile modality due to their high spatial resolution, robustness under varying conditions, and low data volume.
    However, existing approaches typically rely on a single whisker for active control, limiting effectiveness when encountering sharp corners.
    In this work, we introduce a development platform that integrates multiple magnetically transduced whisker sensors on each side, mounted on a robotic arm for active control.
    For single-whisker configurations, we propose an enhanced contour-following strategy paired with an object retrieval policy to manage unavoidable whisker detachments at sharp corners and accurately capture their contours.
    For multi-whisker setups, a tunneling policy is introduced, and all control policies are combined into a single finite state machine that dynamically selects the active policy and determines which whiskers are relied on.
    Simulation results demonstrate that our method effectively tracks objects, reconstructs contours, and maintains a centered trajectory in confined passages.
    Finally, we present a comprehensive framework for real-world experiments, including data collection, preprocessing, evaluation, storage, and visualization.
}{
    Die Rekonstruktion von Objektkonturen ist in der Robotik entscheidend für Aufgaben wie Erkennung, Manipulation und Navigation.
    Vibrissensensoren bieten eine vielversprechende taktile Modalität aufgrund ihrer hohen räumlichen Auflösung, Robustheit unter unterschiedlichen Bedingungen und geringen Datenmenge.
    Bestehende Ansätze verlassen sich jedoch typischerweise auf eine einzelne Vibrisse zur aktiven Steuerung, was ihre Effektivität bei scharfen Ecken einschränkt.
    In dieser Arbeit stellen wir eine Entwicklungsplattform vor, die mehrere magnetisch transduzierte Vibrissensensoren auf jeder Seite integriert und an einem Roboterarm zur aktiven Steuerung montiert ist.
    Für Ein-Vibrissen-Konfigurationen schlagen wir eine verbesserte Konturfolgestrategie in Kombination mit einer Objektrückholungsstrategie vor, um unvermeidbare Vibrissenablösungen an scharfen Ecken zu handhaben und deren Konturen präzise zu erfassen.
    Für Mehr-Vibrissen-Setups wird eine Tunneling-Policy eingeführt, und alle Steuerungsrichtlinien werden in einem einzigen endlichen Automaten zusammengefasst, der dynamisch die aktive Policy auswählt und bestimmt, auf welche Vibrissen sich verlassen wird.
    Simulationsergebnisse zeigen, dass unsere Methode Objekte effektiv verfolgt, Konturen rekonstruiert und in engen Passagen einen zentrierten Kurs beibehält.
    Abschließend präsentieren wir einen umfassenden Rahmen für Realwelt-Experimente, einschließlich Datenerfassung, -vorverarbeitung, Evaluation, Speicherung und Visualisierung.
}
