% In total max. 1 Page!
\AIRstudentthesisAbstract{
    Object contour reconstruction is critical in robotics for tasks such as navigation and recognition.
    Whisker sensors offer a promising tactile modality due to their high spatial resolution, robustness under varying conditions, and low computational requirements.
    However, existing approaches typically fail to reconstruct objects with sharp corners, as the whisker detaches from the object, and do not allow for contour reconstruction in confined spaces.
    In this work, we introduce a development platform that integrates multiple magnetically transduced whisker sensors, mounted on a robotic arm for active control.
    For single-whisker configurations, we propose an enhanced contour-following strategy paired with an object retrieval policy to manage unavoidable whisker detachments at sharp corners and accurately capture their contours.
    For multi-whisker setups, a tunneling policy is introduced, and all control policies are combined into a single finite state machine that dynamically selects the active policy and determines which whiskers are relied on.
    Simulation results demonstrate that our method effectively reconstructs contours with sharp angles, and maintains a centered trajectory in confined passages.
    Finally, we present a comprehensive framework for real-world experiments, including data collection, preprocessing, evaluation, storage, and visualization.
}{
    Die Rekonstruktion von Objektkonturen ist in der Robotik für Aufgaben wie Navigation und Erkennung von entscheidender Bedeutung.
    Schnurrbart-Sensoren bieten aufgrund ihrer hohen räumlichen Auflösung, Robustheit unter wechselnden Bedingungen und geringem Rechenaufwand eine vielversprechende taktile Modalität.
    Bestehende Ansätze scheitern jedoch typischerweise daran, Objekte mit scharfen Ecken zu rekonstruieren, da sich der Schnurrbart vom Objekt löst, und erlauben keine Konturrekonstruktion in engen Räumen.
    In dieser Arbeit stellen wir eine Entwicklungsplattform vor, die mehrere magnetisch transduzierte Schnurrbart-Sensoren integriert, die auf einem Roboterarm für aktive Steuerung montiert sind.
    Für Einzel-Schnurrbart-Konfigurationen schlagen wir eine verbesserte Konturverfolgungsstrategie in Kombination mit einer Objekt-Rückholpolitik vor, um unvermeidliche Schnurrbartablösungen an scharfen Ecken zu handhaben und deren Konturen präzise zu erfassen.
    Für Mehr-Schnurrbart-Systeme wird eine Tunneling-Politik eingeführt, und alle Steuerungsstrategien werden in einer einzigen endlichen Zustandsmaschine kombiniert, die dynamisch die aktive Strategie auswählt und bestimmt, auf welche Schnurrbärte vertraut wird.
    Simulationsergebnisse zeigen, dass unsere Methode Konturen mit scharfen Winkeln effektiv rekonstruiert und eine zentrierte Bahn in engen Passagen beibehält.
    Abschließend präsentieren wir ein umfassendes Framework für reale Experimente, einschließlich Datenerfassung, Vorverarbeitung, Evaluierung, Speicherung und Visualisierung.
}
