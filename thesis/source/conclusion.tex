% !TeX spellcheck = en_US


\chapter{Conclusion}


\section{Summary}

In this work, the groundwork is laid for achieving \textbf{autonomous active tactile exploration of unstructured environments using a whisker array}.
\textbf{Four specialized control policies} are implemented to steer the whisker array platform for contour reconstruction and tunnel navigation.
A \textbf{simulation framework backed by MuJoCo} is established to evaluate the whisker control system.

All policies demonstrate the ability to capture contours with \textbf{millimeter-range precision} while navigating complex environments.
The swiping policy reconstructs object contours while preserving exploration direction and maintaining whisker contact.
It is shown that the swiping policy is able to reconstruct smooth contours and \textbf{overcome angles of up to 30\degree{} without detachment}.
The retrieval policy is proposed to handle whisker detachment at sharp corners.
It is shown that the retrieval policy successfully \textbf{reconstructs object edges} and enables smooth whisker detachment handling, ensuring complete contour capture.
For navigating confined spaces, the tunneling policy is presented.
Multiple test scenarios show the tunneling policy successfully \textbf{navigates tunnels with a diameter of 22\,cm, effectively managing whisker reattachments at sharp corners}.
Furthermore, a governing policy is proposed to \textbf{handle transitions between exploration, swiping, retrieval, and tunneling policies}.
Its efficacy is demonstrated clearly in scenarios such as the zigzag tunnel, where brief whisker detachment occurs, and in all tests involving the retrieval policy.

For testing in real-world scenarios, the \textbf{whisker array platform} is designed and assembled, supporting three whiskers on each side.
It can be mounted on the gripper of the Franka Emika Panda robotic arm, which acts as an actuator for the whisker array.
Finally, a \textbf{system infrastructure} for real-time sensor data visualization and evaluation is developed.
A \textbf{dashboard} is set up to visualize whisker deflection data and the platform's position.


\section{Future Work}

\subsection{Next Steps}
\begin{enumerate}
    \item It is essential to test the whisker platform not only in simulation but also in \textbf{real-world scenarios}.
    Since the platform is already assembled, the next step is testing the control system with the platform attached to the Franka Emika Panda robotic arm.
    The arm acts as an actuator, moving the whisker array platform based on evaluated control policies.
    Real-world experiments will differ from simulations due to \textbf{friction at contact points} between whiskers and objects, as well as \textbf{noise from surface textures} affecting deflection measurements.
    Another critical factor is the behavior of nitinol whiskers under axial loads.
    This is especially relevant for the tunneling policy, where whiskers often collide head-on with surfaces, causing the tip to become stuck and resulting in unpredictable deflections.

    \item The whisker \textbf{deflection model must be recalibrated} separately for (a) simulations and (b) real-world environments.
    In simulations, using ground truth whisker tip positions can simplify tuning the control policies.
    End-to-end control tests provide a realistic evaluation of the entire system but complicate debugging, as component errors propagate through the system.
    The current deflection model accurately covers only a limited deflection range, which is insufficient for the tunneling policy.
    The platform may be placed in tight spaces, leading to large whisker \textbf{deflections exceeding the model’s valid range}.
    Significant deviations were found during testing of the tunneling policy, even in simple scenarios, with predicted whisker tip positions typically closer to the platform than their actual positions.
    Recalibrating the deflection model will improve contour reconstruction performance by increasing whisker tip position accuracy.
\end{enumerate}

\subsection{Potential Improvements}
\begin{itemize}
    \item An \textbf{advanced whisker deflection model} is required to detect contact along the entire whisker length.
    This involves processing additional axes of the hall sensor.
    For this, more advanced suspensions might be required, which \textbf{allow both rotation and axial displacement of the whisker}.

    \item \textbf{More whiskers} can be included in the contour reconstruction process.
    Currently, the control policies evaluate only a maximum of one whisker per side.
    Using even two whiskers per side would significantly speed up contour reconstruction, eliminating the need for swiping back during retrieval.
    The platform dimensions allow the rear-most whisker to seamlessly capture the edge contour.
    The retrieval policy would then be limited to resolving the encountered edge angles.

    \item \textbf{Variable speed} of the whisker array platform is required to enable faster exploration.
    Currently, the platform moves at a constant speed of 5\,cm/s, except during the retrieval policy’s repositioning phase, when it is allowed to rotate in place.

    \item An \textbf{advanced exploration policy and environment mapping} capability is required for the whisker array to achieve SLAM in unstructured environments.
    This involves discovering new objects and determining optimal exploration paths.

    \item For a robust tunneling policy, \textbf{collision detection for the platform} is necessary.
    Currently, collisions involving the platform’s nose, frequently occurring in confined spaces, are not detected.
    As a result, exploration halts since the whisker array cannot move to the desired position.

    \item For 3D contour reconstruction, a \textbf{PCB with multiple hall sensors} is required.
    The simplest arrangement would involve three whisker sensors positioned in a triangle with side lengths significantly shorter than the whisker length.
    Such an arrangement is currently not achievable with Adafruit MLX90393 development boards due to their large size.

    \item Integrating the whisker array into the \textbf{robotic rat} would represent the pinnacle of the project.
\end{itemize}
