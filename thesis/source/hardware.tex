% !TeX spellcheck = en_US


\chapter{Hardware Design}


\section{Magnetically Transduced Whisker Sensor}
The structure of a single whisker sensor is shown in \cref{fig:whisker_composite}.

\begin{figure}[ht]
    \centering
    \begin{subfigure}[b]{0.31\textwidth}
        \centering
        \includegraphics[height=0.2\textheight]{figures/whisker}
        \caption{Whisker mounted on the suspension} \label{fig:whisker}
    \end{subfigure}
    \hspace*{\fill}
    \begin{subfigure}[b]{0.31\textwidth}
        \centering
        \includegraphics[width=\linewidth]{figures/suspension}
        \caption{Suspension design} \label{fig:suspension}
    \end{subfigure}
    \hspace*{\fill}
    \begin{subfigure}[b]{0.31\textwidth}
        \centering
        \includegraphics[width=\linewidth]{figures/whisker-dims}
        \caption{Structural dimensions} \label{fig:whisker-dims}
    \end{subfigure}
    \caption{Single whisker sensor.}
\end{figure}

The whisker consists of:
\begin{itemize}
    \item A flexible nitinol wire shaft (0.25\,mm diameter, 75\,mm length).
    \item A suspension system fabricated via 3D printing using PLA filament.
    \item An Adafruit MLX90393 magnetic sensor, configured to measure magnetic flux changes with a resolution of 0.15\,$\mu T$/LSB.
\end{itemize}

The whisker shaft is inserted into and glued to the suspension system.
A neodymium permanent magnet, axially magnetized and aligned with the whisker shaft, is placed directly beneath the suspension.
When the whisker deflects, the magnet rotates, altering the sensed magnetic field.

The suspension is designed to allow slight rotation of the whisker shaft while limiting axial movement.
Three spring-like arms hold the whisker securely in place.


\section{Whisker Platform}

A whisker platform was developed to enable multi-whisker exploration and contour reconstruction.
It consists of a triangular body with a 30\degree{} nose angle, two side clamps that each hold up to three whiskers, and a mount compatible with the Franka Panda robotic arm gripper.
The platform, depicted in \cref{fig:platform}, measures 90\,mm $\times$ 60\,mm $\times$ 35\,mm and is 3D printed using PLA filament.
Figure~\ref{fig:whisker_platform} shows the printed and assembled platform, designed to roughly resemble a rat’s shape.

The MLX90393 sensors are secured by side clamps designed with cuts to accommodate 4-pin JST connectors on both sides.
All components are assembled using M2 and M3 screws.

The clamps and suspension bases are designed to allow whiskers to be quickly plugged into place and locked by slight rotation, enabling easy replacement.
No external power supply beyond the JST connector is necessary.

Whiskers are mounted at a 15\degree{} angle relative to the platform’s mirror plane, slightly pointing forward to enlarge the contact search area.
However, this orientation poses the risk of head-on collisions during tunneling, potentially causing excessive deflection, permanent deformation, or breaking.
Implementing safeguards within the control system to mitigate this is beyond the scope of this thesis.
Thus, in simulations, whiskers are oriented slightly backward to avoid these issues.

\begin{figure}[ht]
    \centering
    \begin{subfigure}[b]{0.45\textwidth}
        \centering
        \includegraphics[height=0.3\textheight]{figures/platform-cad}
        \caption{Platform CAD model.}
    \end{subfigure}
    \hfill
    \begin{subfigure}[b]{0.45\textwidth}
        \centering
        \includegraphics[height=0.3\textheight]{figures/platform-sketch}
        \caption{Platform sketch from below.}
    \end{subfigure}
    \caption{Whisker platform CAD model and sketch.}
    \label{fig:platform}
\end{figure}

\begin{figure}[htb]
    \centering
    \includegraphics[width=\textwidth]{figures/platform}
    \caption{Assembled whisker platform with three whiskers and robotic arm mount.}
    \label{fig:whisker_platform}
\end{figure}


\section{Data Acquisition}

The Adafruit MLX90393 development board (\cref{fig:sensor}), which hosts a Hall sensor, is placed approximately 2--3\,mm beneath the suspension-mounted magnet.
The sensor utilizes a 16-bit ADC providing magnetic flux density measurements along X, Y, and Z axes.
It is configured to measure with a resolution of \(0.15\,\mu T/LSB\).
In practice, only the Z-axis component is used, as it uniquely captures rotation caused by planar whisker deflections.
The sensors communicate via the I2C protocol, operating in slave mode.
When multiple sensors are used, they are connected in a daisy-chain arrangement.

Two drivers have been developed for the MLX90393 sensors: one in C++ for the ESP32 microcontroller and another in Python for the Raspberry Pi 5.
Both implementations follow the official MLX90393 Triaxis® Magnetic Node specification~\cite{MLX90393}.
Continuous burst mode is employed, reading sensor data at regular intervals, allowing a data acquisition rate of 300\,Hz.
This high rate facilitates effective signal filtering, especially during rapid whisker movements.
The control loop itself operates at approximately 30\,Hz, constrained by the actuator’s responsiveness.

\begin{figure}[ht]
    \centering
    \begin{subfigure}[b]{0.45\textwidth}
        \centering
        \includegraphics[width=\linewidth]{figures/mlx90393}
        \caption{Adafruit MLX90393 devboard (25.4\,mm $\times$ 17.8\,mm $\times$ 1.6\,mm), from~\cite{MLX90393}}
    \end{subfigure}
    \hfill
    \begin{subfigure}[b]{0.45\textwidth}
        \centering
        \includegraphics[width=\linewidth]{figures/mlx90393-chip}
        \caption{Melexis MLX90393 chip (3.1\,mm $\times$ 2.9\,mm $\times$ 1\,mm), from~\cite{MLX90393}}
    \end{subfigure}
    \caption{MLX90393 sensor hardware.}
    \label{fig:sensor}
\end{figure}

A calibrated deflection model converts magnetic field strength readings into whisker deflection angles.
This model is generated by sampling sensor output at different deflection angles and fitting a fifth-degree polynomial to the collected data.
While this polynomial fit is less precise outside the intended operational range, it sufficiently covers the typical operational deflection range ($-5\times10^{-4}$ to $5\times10^{-4}$\,rad).
Figure~\ref{fig:deflection_profile} illustrates this deflection profile, which accounts for offset from the neutral position and stabilizes outputs by clipping values exceeding realistic deflection limits.

\begin{figure}[htb]
    \centering
    \includegraphics[width=0.8\textwidth]{figures/deflection_profile}
    \caption{Whisker deflection profile model.}
    \label{fig:deflection_profile}
\end{figure}
