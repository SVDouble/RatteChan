% !TeX spellcheck = en_US
\documentclass[AIRbeamer
,optEnglish
%,handout%               deactivate animation
,optBiber
,optBibstyleAlphabetic
,optBeamerClassicFormat% 4:3 format
%,optBeamerWideFormat%   16:9 format
]{AIRlatex}

\usepackage{algorithm}
\usepackage{algorithmicx}

%\setbeameroption{show notes}

\graphicspath{{figures/}}%
\addbibresource{source/literature.bib}%

\title[Active Tactile Exploration Based on Whisker-Inspired Sensory Array]{Active Tactile Exploration Based on Whisker-Inspired Sensory Array}
\def\PresentationType{\AIRlangGerEng{Abschlussvortrag}{Final Presentation}}
\def\PresentationThesisType{\AIRlangBachelorsThesis}
\author[Valentin Safronov]{Valentin Safronov}
\def\PresentationExaminer{\AIRnamesProfKnoll}
\def\PresentationSupervisor{Yixuan Dang, M.Sc.}
\date{\AIRutilsDate{28}{03}{2025}}

\AIRbeamerSetupHeader{\AIRlayoutHeaderCustomChair}
\AIRbeamerSetupFooterCD

\begin{document}
    \AIRbeamerTitlePageStudentThesis
    \AIRbeamerSetFooterText{\AIRutilsDate{28}{03}{2025}}

    \begin{frame}{Outline}
        \tableofcontents
    \end{frame}


    \section{Motivation}

    \subsection{Biological Whisker Structure}
    \begin{frame}{Biological Whisker Structure}
        - image of a cat's whisker
    \end{frame}

    \subsection{Robotic Whisker Sensors}
    \begin{frame}{Robotic Whisker Sensors}
        - different types of robotic whisker sensors
    \end{frame}


    \section{Related Work}
    \begin{frame}{Similar Work}
        - overview of the related work
    \end{frame}


    \section{Hardware}

    \subsection{Single Whisker Sensor}

    \begin{frame}[c]{Magnetically Transduced Whisker Sensor}
        \centering
        \begin{columns}[c,onlytextwidth]
            \column{0.3\textwidth}
            \centering
            \includegraphics[height=0.6\textheight]{figures/whisker}\\
            (a) Whisker Sensor,\\Whisker shaft -- a nitinol wire
            \column{0.3\textwidth}
            \centering
            \includegraphics[height=0.6\textheight]{figures/suspension}\\
            (b) Suspension,\\3D-printed with PLA
            \column{0.3\textwidth}
            \centering
            \includegraphics[height=0.6\textheight]{figures/whisker-dims}\\
            (c) Whisker sensor dimensions
        \end{columns}
    \end{frame}

    \note[itemize]{
        \item The whisker shaft is first glued to the suspension system.
        \item A neodymium permanent magnet, axially magnetized with its field direction aligned with the wire, is placed underneath.
        \item The suspension hooks are designed to allow screwing in the whiskers into the whisker mount.
    }

    \begin{frame}{Magnetic Sensor}
    \end{frame}

    \subsection{Whisker Sensor Array}

    \begin{frame}{Whisker Platform}
        \begin{figure}[ht]
            \centering
            \includegraphics[width=0.8\textheight]{figures/platform}
            \caption{Three whisker attached to the left side and robotic arm mounted at the top}
        \end{figure}
    \end{frame}


    \section{Control Algorithms: Theory and Practice}

    \subsection{Problem Statement}
    \begin{frame}[c]{Problem Statement}
        \begin{columns}[T,onlytextwidth]
            \begin{column}[T]{0.4\textwidth}
                \only<1-3>{
                    \textbf{Goals:}
                    \begin{itemize}
                        \item Capturing of the whole object
                        \item Precise contour reconstruction
                        \item Navigation in cluttered environments
                    \end{itemize}
                }
                \only<2-3>{
                    \textbf{Assumptions:}
                    \begin{itemize}
                        \item Navigation and reconstruction in 2D
                        \item Rigid stationary objects
                        \item Contact at the tip
                        \item Known platform position
                    \end{itemize}
                }
            \end{column}
            \only<3>{
                \begin{column}[T]{0.2\textwidth}
                    \begin{minipage}[c][.8\textheight][c]{\linewidth}
                        \tikz[baseline=-\baselineskip]\draw[ultra thick,->] (0,0) -- ++ (2,0);
                    \end{minipage}
                \end{column}
                \begin{column}[T]{0.4\textwidth}
                    \begin{minipage}[c][.8\textheight][c]{\linewidth}
                        \textbf{Policies:}
                        \begin{itemize}
                            \item Swiping Policy
                            \item Retrieval Policy
                            \item Tunneling Policy
                            \item Governing Policy
                        \end{itemize}
                    \end{minipage}
                \end{column}
            }
        \end{columns}
    \end{frame}

    \begin{frame}{Body Motion}{Control Algorithm}
        \begin{algorithm}[H]
            \caption{Steer the Platform to Target Position and Orientation}
            \begin{algorithmic}[1]
                \State Require \(^{\mathrm{w}}\boldsymbol{r}^{t+1}\), \(\;^{\mathrm{w}}\alpha^{t+1}\)
                \State \(^{\mathrm{w}}\omega^{t+1} \gets \mathrm{PID}(\;^{\mathrm{w}}\alpha^{t+1} - \;^{\mathrm{w}}\alpha^{t})\)
                \State \(^{\mathrm{w}}\boldsymbol{v}^{t+1} \gets v_{\mathrm{total}} \cdot \dfrac{^{\mathrm{w}}\boldsymbol{r}^{t+1}}{\|^{\mathrm{w}}\boldsymbol{r}^{t+1}\|}\)
                \State Return \(^{\mathrm{w}}\boldsymbol{v}^{t+1}\), \(^{\mathrm{w}}\omega^{t+1}\)
            \end{algorithmic}
            \label{alg:steer_platform}
        \end{algorithm}

        \begin{algorithm}[H]
            \caption{Steer Whisker to Target Position and Orientation}
            \begin{algorithmic}[1]
                \State Require \(^{\mathrm{w}}\boldsymbol{r}_{\mathrm{wsk}}^{t+1}\), \(^{\mathrm{w}}\alpha_{\mathrm{wsk}}^{t+1}\)
                \State \((^{\mathrm{w}}\boldsymbol{v}^{t+1},\, ^{\mathrm{w}}\omega^{t+1}) \gets \mathrm{steer\_body}(\;^{\mathrm{w}}\boldsymbol{r}_{\mathrm{wsk}}^{t+1},\, ^{\mathrm{w}}\alpha_{\mathrm{wsk}}^{t+1})\)
                \State \(^{\mathrm{w}}\boldsymbol{r}_{\mathrm{corr}} \gets [0,\,0,\,^{\mathrm{w}}\omega^{t+1}] \times \boldsymbol{r}_{\mathrm{wsk, body}}\)
                \State \(^{\mathrm{w}}\boldsymbol{v}^{t+1} \gets v_{\mathrm{total}} \cdot \dfrac{^{\mathrm{w}}\boldsymbol{v}^{t+1} + \,^{\mathrm{w}}\boldsymbol{r}_{\mathrm{corr}}}{\|^{\mathrm{w}}\boldsymbol{v}^{t+1} + \,^{\mathrm{w}}\boldsymbol{r}_{\mathrm{corr}}\|}\)
                \State Return \(^{\mathrm{w}}\boldsymbol{v}^{t+1}\), \(^{\mathrm{w}}\omega^{t+1}\)
            \end{algorithmic}
            \label{alg:steer_whisker}
        \end{algorithm}

    \end{frame}

    \subsection{Swiping Policy}
    \begin{frame}{Swiping Policy}{Control Algorithm}
        \begin{algorithm}[H]
            \caption{Swiping Policy}
            \begin{algorithmic}[1]
                \only<1>
                {
                    \State If \(|\delta_{\mathrm{wsk}}^{t}| < \delta_{\mathrm{wsk, thr}}\)
                    Then
                    \State \quad Return \(\;^{\mathrm{w}}\boldsymbol{v}^{t}\), \(\;^{\mathrm{w}}\omega^{t}\)
                    \State End If
                    \State
                    \State \(\;^{\mathrm{s}}\boldsymbol{r}_{\mathrm{tip}}^{t} \gets \mathrm{wsk.defl\_model}(\delta_{\mathrm{wsk}}^{t})\)
                    \State \(\;^{\mathrm{w}}\boldsymbol{r}_{\mathrm{tip}}^{t} \gets \;^{\mathrm{w}}\boldsymbol{r}^{t} + \;^{\mathrm{w}}\boldsymbol{r}_{\mathrm{wsk, body}} + \boldsymbol{R}_{xy}^{2}(\; ^{\mathrm{w}}\alpha_{\mathrm{wsk}}^{t}) \cdot \;^{\mathrm{s}}\boldsymbol{r}_{\mathrm{tip}}^{t}\)
                    \State \(\mathrm{wsk.spline.add\_keypoint}(\;^{\mathrm{w}}\boldsymbol{r}_{\mathrm{tip}}^{t})\)
                    \State If not \(\mathrm{wsk.spline.has\_enough\_points()}\)
                    Then
                    \State \quad Return \(\;^{\mathrm{w}}\boldsymbol{v}^{t}\), \(\;^{\mathrm{w}}\omega^{t}\)
                    \State End If
                }
                \only<2>
                {
                    \setcounter{ALG@line}{10}
                    \State \(\;^{\mathrm{w}}\boldsymbol{\tau}_{\mathrm{spline}}^{t} \gets \dfrac{\mathrm{wsk.spline}(u\mathord{=}u_{k1}) - \mathrm{wsk.spline}(u\mathord{=}u_{k0})}{\|\mathrm{wsk.spline}(u\mathord{=}u_{k1}) - \mathrm{wsk.spline}(u\mathord{=}u_{k0})\|}\)
                    \State \(\;^{\mathrm{w}}\theta_{\mathrm{spline}}^{t} \gets \mathrm{arctan2}(\;^{\mathrm{w}}\boldsymbol{\tau}_{\mathrm{spline}}^{t})\)
                    \State \(\;^{\mathrm{s}}\boldsymbol{r}_{\mathrm{tip, target}}^{t} \gets \mathrm{wsk.defl\_model}\big(\delta_{\mathrm{wsk, target}} \cdot \operatorname{sgn}(\delta_{\mathrm{wsk}}^{t})\big)\)
                    \State \(\Delta\boldsymbol{r}_{tip}^{t} \gets \boldsymbol{R}_{xy}^{2}(\; ^{\mathrm{w}}\alpha_{\mathrm{wsk}}^{t}) \cdot (\;^{\mathrm{s}}\boldsymbol{r}_{\mathrm{tip, target}}^{t} - \;^{\mathrm{s}}\boldsymbol{r}_{\mathrm{tip}}^{t})\)
                    \State \(\;^{\mathrm{s}}\boldsymbol{r}_{\mathrm{tip, neutral}} \gets \mathrm{wsk.defl\_model}(0)\)
                    \State \(w_{\mathrm{defl}}^{t} \gets \dfrac{\|\Delta\boldsymbol{r}_{\mathrm{tip}}^{t}\|}{\|\;^{\mathrm{s}}\boldsymbol{r}_{\mathrm{tip, target}}^{t} - \;^{\mathrm{s}}\boldsymbol{r}_{\mathrm{tip, neutral}}\|}\)
                    \State \(\;^{\mathrm{w}}\boldsymbol{r}_{\mathrm{wsk}}^{t+1} \gets \;^{\mathrm{w}}\boldsymbol{r}_{\mathrm{wsk}}^{t} + w_{\mathrm{defl}}^{t} \cdot \dfrac{-\Delta\boldsymbol{r}_{\mathrm{tip}}^{t}}{\|\Delta\boldsymbol{r}_{\mathrm{tip}}^{t}\|} + (1 - w_{\mathrm{defl}}^{t}) \cdot \dfrac{\;^{\mathrm{w}}\boldsymbol{\tau}_{\mathrm{spline}}^{t}}{\|\;^{\mathrm{w}}\boldsymbol{\tau}_{\mathrm{spline}}^{t}\|}\)
                    \State \((\;^{\mathrm{w}}\boldsymbol{v}^{t+1}, \;^{\mathrm{w}}\omega^{t+1}) \gets \mathrm{steer\_wsk}(\;^{\mathrm{w}}\boldsymbol{r}_{\mathrm{wsk}}^{t+1},\;^{\mathrm{w}}\theta_{\mathrm{spline}}^{t})\)
                    \State Return \(\;^{\mathrm{w}}\boldsymbol{v}^{t+1}, \;^{\mathrm{w}}\omega^{t+1}\)
                }
            \end{algorithmic}
        \end{algorithm}
    \end{frame}
    \begin{frame}{Swiping Policy}{Simulation Results}
        \begin{figure}[htb]
            \centering
            \includegraphics[width=0.7\textwidth]{figures/experiments/rounded-rectangular-box-swiping}
        \end{figure}
    \end{frame}
    \begin{frame}{Swiping Policy}{Simulation Results}
        \begin{figure}[htb]
            \centering
            \includegraphics[width=0.7\textwidth]{figures/experiments/complex-object-swiping}
        \end{figure}
    \end{frame}

    \subsection{Retrieval Policy}
    \begin{frame}[c]{Retrieval Policy}{Motivation}
        \begin{columns}[T,onlytextwidth]
            \begin{column}[T]{0.48\textwidth}
                \begin{figure}[H]
                    \centering
                    \captionsetup{justification=centering}
                    \includegraphics[width=\textwidth]{figures/retrieval/before}
                    \caption{\textbf{(0.1) Initial situation}\\Swiping along the side of the object.}
                \end{figure}
            \end{column}
            \begin{column}[T]{0.48\textwidth}
                \begin{figure}[H]
                    \centering
                    \captionsetup{justification=centering}
                    \includegraphics[width=\textwidth]{figures/retrieval/disengagement}
                    \caption{\textbf{(0.2) Disengagement}\\Whisker has detached.}
                \end{figure}
            \end{column}
        \end{columns}
    \end{frame}
    \begin{frame}[c]{Retrieval Policy}{Angle Resolution}
        \begin{columns}[T,onlytextwidth]
            \begin{column}[T]{0.48\textwidth}
                \begin{minipage}[c][.8\textheight][c]{\linewidth}
                    \begin{enumerate}
                        \item \textbf{Angle Resolution}
                        \begin{enumerate}
                            \item Construct a circle of potential contact points
                            \item Move to the first candidate point
                            \item Move sequentially from one candidate point to another until the contact is established
                            \item Calculate the edge angle
                        \end{enumerate}
                    \end{enumerate}
                \end{minipage}
            \end{column}
            \begin{column}[T]{0.48\textwidth}
                \begin{figure}[H]
                    \centering
                    \begin{tikzpicture}[scale=2]
                        \coordinate (E) at (0,0);
                        \coordinate (X) at (2,0);
                        \coordinate (Y) at ({2*cos(30)},{2*sin(30)});
                        \coordinate (Yp) at ({2*cos(135)},{2*sin(135)});
                        \coordinate (C) at ({cos(30)},{sin(30)});
                        \coordinate (W) at ({cos(135)}, {sin(135)});
                        \coordinate (T) at ({cos(135) - cos(135 - 90)}, {sin(135) - sin(135 - 90)});
                        \coordinate (B) at ({cos(135) - 2*cos(135 - 90 + 20)}, {sin(135) - 2*sin(135 - 90 + 20)});

                        \draw [blue, dotted] (E) circle (1);
                        \draw (E) -- (X);
                        \draw (E) -- (Y);
                        \draw [green] (B) -- (W);
                        \draw [blue, dashed] (E) -- (Yp);
                        \draw [blue, dashed] (W) -- (T);

                        % Mark the 30° angle at vertex B.
                        \draw (0.5,0) arc (0:30:0.5);
                        \node at (0.8,0.15) {$\beta$};

                        % Label the points.
                        \fill (X) circle (1pt) node[below right] {$X$};
                        \fill (E) circle (1pt) node[below left] {$E$};
                        \fill (Y) circle (1pt) node[above right] {$Y$};
                        \fill (Yp) circle (1pt) node[above right] {$Y'$};
                        \fill (W) circle (1pt) node[above left] {$W$};
                        \fill (B) circle (1pt) node[above left] {$B$};
                        \fill (T) circle (1pt) node[above left] {$T$};

                    \end{tikzpicture}
                    \caption{Angle Resolution at Edge E \\Black $[XE]$, $[EY]$ \textemdash{} object surface,\\Green $[BW]$ \textemdash{} the whisker,\\Blue $[EY']$ \textemdash{} the potential adjacent surface of the edge and its tangent $[WT]$,\\Blue circle \textemdash{} the targeted contact points.}
                \end{figure}
            \end{column}
        \end{columns}
    \end{frame}
    \begin{frame}[c]{Retrieval Policy}{Angle Resolution}
        \begin{columns}[T,onlytextwidth]
            \begin{column}[T]{0.48\textwidth}
                \begin{figure}[H]
                    \centering
                    \captionsetup{justification=centering}
                    \includegraphics[width=\textwidth]{figures/retrieval/retrieval}
                    \caption{\textbf{(1.1) Retrieval}\\Whisker has retrieved the contact.}
                \end{figure}

            \end{column}
            \begin{column}[T]{0.48\textwidth}
                \begin{figure}[H]
                    \centering
                    \captionsetup{justification=centering}
                    \includegraphics[width=\textwidth]{figures/retrieval/retrieval}
                    \caption{\textbf{(1.2) Retrieval}\\Whisker has retrieved the contact.}
                \end{figure}
            \end{column}
        \end{columns}
    \end{frame}
    \begin{frame}[c]{Retrieval Policy}{Whisking and Transition to Swiping}
        \begin{enumerate}
            \setcounter{enumi}{1}
            \item \textbf{Whisking Back to the Edge}
            \begin{enumerate}
                \item Move back along the opposite side of the edge
                \item Disengage the whisker at the edge
            \end{enumerate}
            \item \textbf{Transition to Swiping}
            \begin{enumerate}
                \item Reposition the whisker with a slight overshoot
                \item Move towards the object until the contact is re-established
                \item Transfer control to the exploration policy
            \end{enumerate}
        \end{enumerate}
    \end{frame}
    \begin{frame}[c]{Retrieval Policy}{Whisking and Repositioning}
        \begin{columns}[T,onlytextwidth]
            \begin{column}[T]{0.48\textwidth}
                \begin{figure}[H]
                    \centering
                    \includegraphics[width=\textwidth]{figures/retrieval/whisking}
                    \caption{\textbf{(2) Whisking}\\Moving back along the opposite side of the edge.}
                \end{figure}
            \end{column}
            \begin{column}[T]{0.48\textwidth}
                \begin{figure}[H]
                    \centering
                    \includegraphics[width=\textwidth]{figures/retrieval/repositioning}
                    \caption{\textbf{(3.1) Repositioning}\\Assuming an optimal position for approach.}
                \end{figure}
            \end{column}
        \end{columns}
    \end{frame}
    \begin{frame}[c]{Retrieval Policy}{Final Steps}
        \begin{columns}[T,onlytextwidth]
            \begin{column}[T]{0.48\textwidth}
                \begin{figure}[H]
                    \centering
                    \includegraphics[width=\textwidth]{figures/retrieval/approach}
                    \caption{\textbf{(3.2) Approach}\\Moving towards the object.}
                \end{figure}
            \end{column}
            \begin{column}[T]{0.48\textwidth}
                \begin{figure}[H]
                    \centering
                    \includegraphics[width=\textwidth]{figures/retrieval/trajectory}
                    \caption{\textbf{(3.3) Transition to Swiping}\\Transferring control to the Swiping Policy.}
                \end{figure}
            \end{column}
        \end{columns}
    \end{frame}
    \begin{frame}{Retrieval Policy}{Simulation Results}
        \begin{figure}[htb]
            \centering
            \includegraphics[width=0.7\textwidth]{figures/experiments/octagon-edges-135deg-swiping-retrieval}
        \end{figure}
    \end{frame}
    \begin{frame}{Retrieval Policy}{Simulation Results}
        \begin{figure}[htb]
            \centering
            \includegraphics[width=0.7\textwidth]{figures/experiments/prism-edges-60deg-swiping-retrieval}
        \end{figure}
    \end{frame}
    \begin{frame}{Retrieval Policy}{Simulation Results}
        \begin{figure}[htb]
            \centering
            \includegraphics[width=0.7\textwidth]{figures/experiments/wall-edges-90deg-swiping-retrieval}
        \end{figure}
    \end{frame}

    \subsection{Tunneling Policy}
    \begin{frame}{Swiping Policy Algorithm}
        - code
    \end{frame}
    \begin{frame}{Swiping Policy Experiments}
        - 2-3 examples
    \end{frame}

    \subsection{Governing Policy}
    \begin{frame}{Governing Policy}{Finite State Machine}
        \begin{figure}[H]
            \centering
            \includegraphics[width=0.6\textwidth]{figures/fsm}
            \caption{FSM diagram of the whisker control system.}
        \end{figure}
    \end{frame}


    \section{Infrastructure}
    \begin{frame}{Infrastructure}{System Overview}
        \begin{figure}[H]
            \centering
            \includegraphics[width=\textwidth]{figures/infrastructure-overview}
            \caption{Data flow in the system infrastructure.}
        \end{figure}
    \end{frame}

    \begin{frame}{Infrastructure}{Robot Controller Overview}
        \begin{figure}[H]
            \centering
            \includegraphics[height=0.75\textheight]{figures/infrastructure-robot-controller}
            \caption{Data flow in the robot controller infrastructure.}
        \end{figure}
    \end{frame}

    \begin{frame}{Infrastructure}{Data Visualization}
        \begin{figure}[H]
            \centering
            \includegraphics[width=0.7\textwidth]{figures/grafana}
            \caption{Sensor data visualization in real time using Grafana.}
        \end{figure}
    \end{frame}


    \section{Future Work}
    \begin{frame}{Future Work}
        - testing on a real robotic arm (Franka Panda)
        - SLAM
    \end{frame}


    \section{Conclusion}
    \begin{frame}{Conclusion}
        - summary of the work
    \end{frame}


    \AIRbeamerSetFooterText{References}
    \begin{frame}[allowframebreaks]{References}
        %\sloppy% "Word"-like typesetting in order to improve breaking lines with long URLs/DOIs
        \printbibliography[heading=none]
    \end{frame}%

    \AIRbeamerTitlePageStudentThesis%

\end{document}%
